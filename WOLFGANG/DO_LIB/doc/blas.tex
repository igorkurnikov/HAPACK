\newpage 
\markboth{../include/blas.h}{../include/blas.h}

\vspace{0.1em} \small 
{\ \hrulefill Start of Code \hrulefill} 
\begin{verbatim} 
/*IGN%  Dortmund C++ Class and Template Library 
%  Copyright (C) 1994 Wolfgang Wenzel

%  This library is free software; you can redistribute it and/or
%  modify it under the terms of the GNU Library General Public
%  License as published by the Free Software Foundation; either
%  version 2 of the License, or (at your option) any later version.

%  This library is distributed in the hope that it will be useful,
%  but WITHOUT ANY WARRANTY; without even the implied warranty of
%  MERCHANTABILITY or FITNESS FOR A PARTICULAR PURPOSE.  See the GNU
%  Library General Public License for more details.

%  You should have received a copy of the GNU Library General Public
%  License along with this library; if not, write to the Free
%  Software Foundation, Inc., 675 Mass Ave, Cambridge, MA 02139, USA.
  
%  $Id: blas.tex,v 1.1.1.1 2008/04/08 19:44:31 igor Exp $
*/


\end{verbatim} 
{\ \hrulefill End of Code \hrulefill} \ \normalsize
 
\chapter{BLAS Interface}

\section{Introduction}

In this chapter we describe the interface of the Matrix and Vector
classes with the Basis Linear Algebra Subroutines (BLAS) package.  The
purpose of interfacing with BLAS is to get nearly the maximal possible
speed for various fundamental operations.  The speed up over the
hand-coded equivalents is machine and operation dependent, but can be
up to a factor of $10$!  (For a more complete description of BLAS try
the manual pages of most modern workstations, or NETLIB at
netlib@ornl.gov ).

To activate this option the compiler time flag \name{-DBLAS} should be
used, and the standard library libblas.a should be loaded using
\name{-lblas}.  {\bf Ask Matthew about how this should be done.}

Not all Matrix and Vector functions have been interfaced yet.
Below is a list of those interfaced.  If you want another function
interfaced ask Matthew (mms@pacific.mps.ohio-stae.edu) -- it is not
difficult, just tedious.

\noindent
\name{Vector\& Vector::operator=(const Vector\& v);}  \\
\name{Vector\& Vector::operator+=(const Vector\& v);}  \\
\name{double Vector::operator*(const Vector\& v);}  \\
\name{void Vector::multiply(const Matrix\& mat, const Vector\& vec);}  \\

\noindent
\name{void Matrix::multiply(const Matrix\& mat2, const Matrix\& mat3);}  \\
\name{void Matrix::multiply\_transpose\_1(const Matrix\& mat2, const Matrix\& mat3);}  \\

\noindent
\name{friend Vector operator+(const Vector\& vec1,const Vector\& vec2);}  \\
\name{friend Vector operator-(const Vector\& vec1,const Vector\& vec2);}  \\

\noindent
Further the following functions (not in the BLAS package) have
unrolled-loop implementations. 

\noindent
\name{void Vector::set(const double value);}  \\
\name{Vector\& Vector::operator+=(const double value);}  \\ 
\vspace{0.1em} \small 
{\ \hrulefill Start of Code \hrulefill} 
\begin{verbatim} 


#ifndef BLAS_H
#define BLAS_H

  //  The machine dependent macro definitions.

#ifdef GNU 
#define Mdcopy dcopy_
#define Mdaxpy daxpy_
#define Mddot  ddot_
#define Mdgemv dgemv_

#ifdef SUN     //  IMSL

#define Mdmrrrr dmrrrr_
#define Mdmxtyf dmxtyf_
#define Mdmxytf dmxytf_

#else

#define Mdgemm dgemm_

#endif

#endif

#ifdef XLC
#define Mdcopy dcopy
#define Mdaxpy daxpy
#define Mddot  ddot
#define Mdgemv dgemv
#define Mdgemm dgemm
#endif

  //  Now the BLAS functions.

  extern "C" {

    void Mdcopy( int*, double*, int*, double*, int* );
    void Mdaxpy( int*, double*, double*, int*, double*, int* );
    double Mddot( int*, double*, int*, double*, int* );
    void Mdgemv( char*, int*, int*, double*, double*, int*, 
                        double*, int*, double*, double*, int* );
    void Mdgemm( char*, char*, int*, int*, int*, 
                 double*, double*, int*, double*, int*, double*, double*, int* );

    void Mdmrrrr( int*, int*, double*, int*, 
                  int*, int*, double*, int*, int*, int*, double*, int* );
    void Mdmxtyf( int*, int*, double*, int*, 
                  int*, int*, double*, int*, int*, int*, double*, int* );
    void Mdmxytf( int*, int*, double*, int*, 
                  int*, int*, double*, int*, int*, int*, double*, int* );

  }

  //  The actual implementation

#include "blas_implement.h"

#endif
\end{verbatim} 
