\documentclass[letterpaper]{article}

\usepackage{array}
\usepackage{color}
\usepackage{listings}
\lstloadlanguages{C++}
\lstset{language=C++}

\title{sleap User's Guide}
\author{Wei Zhang\\The Scripps Research Institute\\10550 North Torrey Pines Road\\La Jolla, CA 92037}

\usepackage{latexsym}
\usepackage{mathptmx}
\usepackage{courier}
\usepackage{helvet}

\usepackage{geometry}
\geometry{verbose,letterpaper,tmargin=1in,bmargin=1in,lmargin=1.25in,rmargin=1in}

\begin{document}
\maketitle

\section{Introduction}

\paragraph*{}
  sleap is a rewritten of amber's user interface tleap, This document describes how to obtain, 
install, test and use sleap. 

\section{Obtaining the source code}

\paragraph*{}
  The source code of sleap is located in the CVS repository of amber10, the location is, people can
obtain it via cvs.

\paragraph*{}
  On our first time obtaining the source code, please use the following command:
\begin{lstlisting}
cvs -d:ext:xxx@loyd.scripps.edu:/raid5/cvsroot/ co amber10/src/gleap
\end{lstlisting}
which will generate a new directory amber10/src/gleap containing all the code.

\paragraph*{}
   After you have check out the module, you can use ``cvs update -d'' to get the latest update.
\section{Installation}

   Installation of sleap is standard procedure and is now part of the installation of amber10.
\begin{lstlisting}
                  cd xxx/amber10/src/gleap
                  make
\end{lstlisting}
To compile sleap you need g++ at least version 3.0 (which means g++-2.96 won't work). If everything 
goes well, you will get the executable sleap under the directory xxx/amber11/src/gleap/leapsrc and
also xxx/amber11/bin/. Sleap has been compiled and tested on linux and windows (through cygwin). Please 
report all the errors to amber-developers@scripps.edu.

\section{Test}

\paragraph*{}
   If compilation goes well, you can try to test your build by running test cases under directry amber10/test/sleap:
\begin{lstlisting}
                  cd xxx/amber11/test/sleap
                  make test
\end{lstlisting}
   

\section{Usage}

\subsection{Limitations}
  sleap now have the following limitations:

\paragraph{SaveAmberParm} won't give the identical topology file as tleap does, while the energy 
should be identical.

\paragraph{SolvateDonClip} has not been implemented.

vdw radii they are using.


\subsection{Unsupported Commands}
  The following commands are not going to be implemented, since it is not clear to me why do
they even exist.

\paragraph*{addAtomTypes} It seems to me the only usage of it is designating the hybrid type of 
an atom, which is determined by chemical environment in sleap.

 
\paragraph*{logFile} All the information are dumped to standard output now.


\paragraph*{CreateParmset} doesn't understand why we need this command.
   

 
\subsection{New Commands or New Features of old Command}

The follwing new commands have been introdcued to sleap:

\paragraph{loadsdf} allow users to read mdl's sdf format file.
The grammer of loadsdf is:

\begin{lstlisting}
    unitname = loadsdf filename
\end{lstlisting}


\paragraph{savesdf} allow users to save mdl's sdf format file.
The grammer of savesdf is:
\begin{lstlisting}
    savesdf unitname filename
\end{lstlisting}

\paragraph{loadmol2} can now load molecules have more than one residue.

\paragraph{savemol2} allow user to save tripos mol2 format file.
The garmmer of savemol2 is:
\begin{lstlisting}
    savemol2 unitname filename
\end{lstlisting}

\paragraph{fixbond} assigns bond order automatically. Note that the input molecule
should have only one residue. The grammer is:
\begin{lstlisting}
    fixbond unitname
\end{lstlisting}
There is a test case showing how to use fixbond in amber10/test/sleap/fastbld

\paragraph{addhydr} addes hydrogens to a molecule. The molecule should have only
one residue and have correct bond order assigned. The grammer is:
\begin{lstlisting}
    addhydr unitname
\end{lstlisting}
There is a test case showing how to use addhydr in amber10/test/sleap/fastbld

\paragraph{setpchg} calls antechamber to set partial charge (am1-bcc) and gaff atom type for a
molecule. The molecule should have only one residue. The grammer is:
\begin{lstlisting}
    setpchg unitname
\end{lstlisting}
There is a test case showing how to use setpchg in amber10/test/sleap/fastbld

\paragraph{parmchk} calls parmchk on a molecule to get missing force field parameters and add 
them to system database. The grammer is:
\begin{lstlisting}
    parmchk unitname
\end{lstlisting}

\subsection{New keywords}
    The following new keywords have been introduced to sleap:

\paragraph{echo} if set to "on", the input command will be echoed. This is very useful 
for the construction of test cases.

\paragraph{disulfide} is used to control the behavior of loadpdb on disulfide bonds. 
if disulfide is set to "off", loadpdb will not create disulfide bonds unless they are
specified in the CONECT records; if disulfide is set to "auto", loadpdb will create disulfide
bonds between two sulfur atoms whose distance is less then the value specified by keyword
"disulfcut" (by default the cutoff is 2.2 angstrom); if disulfide is set to "manu", loadpdb
will ask the user if they want to create such a disulfide bond when found such pair of 
sulfur atoms. by default it is set to off.


\paragraph{disulfcut} is used as the cutoff of disulfide bond.


\paragraph{fastbld} is used to control the behavior of loadpdb upon the unknown residues.
if fastbld was set to "on" and an unknown residue was encoutered in the pdb file, loadpdb 
will try to run fixbond, addhydr, setpchg and parmchk on the unknown residue and put all the
the necessary information together into the molecule. The resulted molecule will then be
ready for SaveAmberParm.


\subsection{The basic idea behind the new commands}
    As has been mentioned before, quite a few new commands have been introduced to sleap. The
utilmate goal of these new commands is that users will be able to generate topology file right
from pdb files without calling any other programs such as antechamber. 
    The easiest way to prepare topology from pdb file is to use the new keyword fastbld. Ideally
the script would look like the following:

\begin{lstlisting}

     source leaprc.ff03
     source leaprc.gaff
     set default fastbld on
     xxx = loadpdb xxx.pdb
     saveamberparm xxx xxx.top xxx.xyz
     quit

\end{lstlisting}

    However, the real world cases can not always be that simple. There are several issues which
could interrrupt the procedure. 

    Firstly, the fixbond command could fail on distorted structures. Fixbond uses the geometrical
evidences to determine the bond orders. It won't work for distorted structures.

    Secondly, the addhydr command might not give the proper answer since it does not consider
protonate states.

    Thirdly, the setpchg command only assign am1-bcc charge to the residue. Sometimes users might
want to use the resp charge.

    In all, experienced users might want to customize the procedure. They might use some of the
new commands but not all of them. That is the reason the separate commands are provided. The script
in my mind would be something like the following:

\begin{lstlisting}

    source leaprc.ff03
    source leaprc.gaff
    res = loadpdb res.pdb
    fixbond res
    addhydr res
    setpchg res
    parmchk res
    all = loadpdb all.pdb
    savemaberparm all all.top all.xyz
    quit

\end{lstlisting}

   Users may make changes to this script. For instance, one can assign the bond order manually save
the result in sdf format (or mol2 format), then reload in sleap and do the rest, or he can even add
hydrogens by himself. In all, it is a highly customizable procedure.






\end{document}
